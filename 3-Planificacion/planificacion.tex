	\documentclass[../pfc.tex]{subfiles}
	
	\begin{document}
	
	\section{Plan de desarrollo de software (revisar la fase de inicio lo de Mariete)}
	
	Para el desarrollo de este aplicativo se seguirán los principios del Agile Manifesto \cite{agile}, y de entre todas las metodologías que lo implementan utilizaremos Scrum, ya que nos permite desarrollar siempre sobre algo ejecutable y tiene una buena "pelea contra el tiempo" o "timeboxing" ya que al final de cada sprint hay que hacer una retrospectiva sobre lo que ya se ha construido y entregado.

	\section{Propósito general de la planificación}
	
	La planificación nos debe dar una cifra orientativa del esfuerzo a comprometer para acometer un desarrollo de un proyecto software. Pero debido a lo mencionado anteriormente sobre el manifiesto ágil, creemos que dar una cifra estimativa en tiempo es venturoso, más si queremos ceñirnos a el y más aún cuanto mas a largo plazo sea la estimación. Por eso las metodologías ágiles suelen ocultar la referencia temporal de los desarrollos y estiman la complejidad de las tareas, sabiendo por el histórico, ya que los equipos deben ser fijos en el tiempo, la complejidad aproximada que un equipo dedicado a un proyecto puede acometer. 
	
	\section{Scrum, sprints, estimaciones,... }
	
	Para la realización del proyecto hemos elegido dentro las diferentes metodologías ágiles SCRUM, por ser la que mejor se adapta a la continua pelea contra el tiempo que el equipo debe mantener. 
	Lo primero que debemos decir es que SCRUM es una metodología iterativa e incremental que promueve la auto organización de los equipos de desarrollo y un esquema de colaboración con el cliente, haciéndose responsable de que las prioridades, los requisitos, etc pueden cambiar por parte de este, y que es responsabilidad del propio equipo el asumir y responder a estos cambios. 
	
	
	\section{Roles y responsabilidades}
	
	\section{Planificación completa}
	
	\section{Versionado}
	
	Esto para que es?
	
	\section{Recursos necesarios}


	
\end{document}