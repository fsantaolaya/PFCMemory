\documentclass[../pfc.tex]{subfiles}

\begin{document}
	
	En este capítulo se presentan las conclusiones obtenidas de la realización de este proyecto,las dificultades encontradas, la consecución de objetivos marcados en un principio, los conocimientos adquiridos y las propuestas para trabajos\\* futuros.\\*
	
	\section{Conclusiones}
	
	Después de la realización de este proyecto vemos que se han conseguido los objetivos fijados en un principio.\\*
	
	En este proyecto hemos utilizado las muchas prácticas y disciplinas de varias de las metodologias agiles para llevar a cabo el desarrollo de un producto software.\\* 
	
	En ese sentido debemos decir que como dice el paper de Fred Brooks{{citisismaaaaaaaa}} "There's no silver bullet" que más o menos dice que no hay nada que magicamente cambie tanto tu forma ed trabajo u organización que leve tu productividad como por arte de magia. Como efecto colateral decir que son las metodologias las que hay que adaptar a las organizaciones. No hay nada que debas o no debas hacer obliugatoriamente sin una buena razón.\\*
	
	En este proyecto hemos desarrollado una aplicación Android coherente con la iniciativa de la AECC 'DIARIO DE SALUD PARA SUPERVIVIENTES DE CÁNCER' y que hace más fácil llevar el control de muchas de las actividades relacionadas con la supervisión del tratamiento una vez 'superada' la enfermedad.\\*
	
	La aplicación posibilita el control de las citas médicas, rutinas diarias del paciente, control de la medicación, registro de síntomas y personas relacionadas con el paciente así como la gestión de pruebas médicas y su posterior uso informativo por parte del usuario.\\*
	
	La aplicación Android 'Diario de un superviviente' es pionera en el control total del tratamiento de un paciente de cáncer, con lo cual, no existe nada parecido en el mercado Android, y esto nos permite abrir una brecha e iniciar el camino en el desarrollo de aplicaciones destinadas a mejorar y sobrellevar el tratamiento de una de las principales causas de muerte en el mundo actual de la mano y contando con la inestimable experiencia de la AECC.\\*
	
	El diseño del proyecto se centró en el objetivo principal que era proporcionar una base de control fiable de las citas médicas de un paciente, facilitando su preparación y recopilación de información por parte del paciente para aprovechar mejor la visita al especialista y hacer más aprovechable la misma, dicho en otras palabras 'Para que no se escape ningún detalle'.\\*
	
	Al usuario además se le anima a realizar actividades diarias rutinarias, pudiendo anotar el grado de satisfacción que es tas le provocan.\\*
	
	Una vez que se consiguió la realización del principal objetivo, se vio necesario demostrar la utilidad de la aplicación en un entorno 'real' con pruebas de un uso diario intensivo por parte de un paciente para así recopilar información acerca de su facilidad de uso y detectar posibles carencias en la misma, esto propició la aplicación de mejoras y el control de errores por nuestra parte.\\*
	
	Por otra parte, la inserción de medicamentos, pruebas, síntomas y personajes, ayudó a que la información fuese más completa y estuviese más integrada en la aplicación, permitiendo añadir funcionalidades a la misma, y abriendo un abanico de posibilidades a la hora de tener que ampliar funcionalidades.\\*
		
	Otra conclusión, esta vez algo más negativa, es la que nos lleva a pensar en el porqué de esta escasez de esfuerzos a la hora de proporcionar herramientas tan útiles que ayuden a mejorar la situación diaria de los pacientes, y que a buen seguro, asociaciones medicas y de enfermos recibirían con los brazos abiertos.\\*
	 
		
	Por otra parte, y no siendo menos importante, se nos da la posibilidad de luchar contra la enfermedad desde otro frente, que es desde el punto de vista del paciente, que es el que sufre la devastación de los tratamientos y las penalidades asociadas al mismo y que dada la cercanía desde la que nos toca el tema nos llena de orgullo.\\*
	
	Como punto final de las conclusiones, indicamos que el desarrollo de la aplicación continuará, ya sea por nosotros o por otros profesionales, para dar continuidad a esta herramienta y mejorarla hasta donde sea posible, alargando la colaboración junto a la AECC el tiempo que sea posible.\\*
	
	

	\section{Dificultades encontradas}
	
	Una de las dificultades encontradas, fue la de idear todo el diseño de la aplicación, ya que al no haber ejemplos existentes no teníamos ninguna referencia, con lo cual, y utilizando muchas de las directrices dadas por Google con Material design pudimos solventar no sin poco esfuerzo.\\*

	Una de las mayores dificultades la ha constituido el trato directo con el cliente. Entendemos ahora que exista un perfil dedicado a esta tarea que sea capaz de comprender los dos idiomas, el del cliente y el del equipo de desarrollo, y que ejerza de lubricante enlas relaciones entre los mismos. Al tener que doblar el papel en el proyecto hemos visto la cantidad de tiempo que puede invertirse en que una funcionalidad este claramente definida en su comportamiento y relacion con el resto de la aplicación. Dado que ninguno de los alumnos que realizan este proyecto tiene un perfil con trato directo y atención al cliente, ahora valoramos mucho este tipo de perfiles que liberan de presion y atención al equipo, permitiendo que este esté centrado en el desarrollo tecnico de lo que el cliente quiere. 		
	 
	Una dificultad más que encontramos, ha sido la de realizar la documentación de una manera profesional y con un control total por nuestra parte, para ello desechamos programas propietarios y nos decidimos por LateX\cite{sharelatex} junto a TexStudio\cite{texstudio} para la realización de la memoria, que si bien tiene una curva de aprendizaje pronunciada, posteriormente nos ha facilitado bastante las cosas a la hora de incluir figuras y tablas y de dar forma a al documento final.\\*
	
		
	\section{Consecución de Objetivos}
	
	El objetivo principal del proyecto desde el principio ha sido el de servir de apoyo y mejorar la iniciativa de la AECC 'Diario de salud para supervivientes de Cáncer', esto se ha conseguido prestando especial atención a las carencias de este programa, lo cuales pueden ser identificados como la rigidez que ofrece el sistema del libro, el dialogo nulo que tiene con el usuario y obviamente la limitación de espacio que supone un libro de unas 30 páginas respecto a la aplicación.\\*
	
	Las listas, las diferentes ordenaciones, el tratamiento individual y pormenorizado de pruebas, síntomas, notificaciones, el poder añadir archivos gráficos y los recursos que ofrece la aplicación hacen de esta una potente herramienta al servicio del paciente.\\*
		
	
	\section{Conocimientos adquiridos}
	
	El haber realizado este proyecto entre dos personas ha sido interesante en el sentido de la cooperación y coordinación de un equipo pequeño para el desarrollo de un proyecto.\\*
	
	Actualmente, los equipos pequeños de 2-3 personas son tendencia a la hora de utilizar una metodología ágil y las decisiones respecto al diseño, desarrollo, funcionalidades, son más rápidas y se llevan a cabo antes.\\*
	
	Hemos aprendido a diseñar una aplicación desde cero, a manejar tiempos y fechas limite, a 'teletrabajar'.\\*
	
	Así mismo hemos aprendido Android, sql, a documentar a través de LateX, a redactar una memoria, así como bastante Photoshop y  versionado a través de Git,\\*
		
	
	\section{Trabajo futuro}
	
	Como posibles mejoras futuras para la aplicación se nos han planteado varias funcionalidades que ya sean por tiempo o por objetivos no se han incluido en esta versión:\\*	
	
	Búsquedas dentro de los listados e incluso búsqueda general dentro de la aplicación que facilite encontrar cualquiera de los tipos persistentes de manera rápida una vez su número haya aumentado de manera considerable.\\*
	
	Estadísticas personalizadas y generales, para determinar comportamientos, hábitos,y que estas puedan desnormalizarse para que no pueda rastrearse los datos personales de las personas pero que si puedan ponerse en estudio de quienes necesiten este tipo de datos en futuros experimentos, etc \\*
	
	Posibilidad de exportar las citas y rutinas a diferentes aplicaciones de sincronización de tareas, tales como facebook, Google, y su posibilidad de recibir correos electrónicos y notificaciones como el resultado de sincronizar con servicios externos para ser visualizadas en otros entornos y por otras personas, haciendo así posible su compartición o utilización por parte de otros servicios.\\*
	
	Parte servidora que permita un dialogo entre la asociación y el usuario más directo, descubriendo un sinfín de posibilidades a nivel de ayuda, cuidados, respuestas y servicios personalizados.\\*
	
	Posibilidad de interactuar de manera directa con un agente de la asociación a través de la aplicación, asignación de un agente a modo de 'ángel de la guarda' .\\*
	
	Poder variar las notificaciones para que se hagan de manera repetitiva, por ejemplo repetir de manera semanal los jueves a las 8 una actividad, es decir generar una rutina más real. \\*
	
	Poder exportar e importar de Google maps, conectar la ubicación con Google maps para poder ir a la cita medica, ya que estas a menudo no son en la misma ciudad.\\*
	
	Llevar el stock de lo que lleva consumido del medicamento el paciente y generar alertas para cuando se le acabe, duplicar tratamiento, para ahorrar trabajo, opciones de dosis.\\*
	
	Llevarnos ese contacto a la agenda o traernosle de ella (PERSONAJES).\\*
	
	Poder utilizar documentos tipo PDF para no depender de la fotografía solo a la hora de adjuntar pruebas médicas.\\*
	
	Posibilidad de dar feedback.\\*
	
	En la sección de recursos, poder editar los canales RSS, chat entre supervivientes, personalizar la canción utilizada para la meditación.\\*
	
	Poder editar los teléfonos que se ofrecen de interés en la sección, para hacerlo más personalizable.\\*
	
	En los ajustes de la aplicación poder cambiar el tono de aviso, individualmente.\\* 
	
	
		
	
	


	

	
\end{document}