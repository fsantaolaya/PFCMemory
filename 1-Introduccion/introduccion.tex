\documentclass[../pfc.tex]{subfiles}

\begin{document}

\section{Antecedentes y Motivación del proyecto}
El cáncer es una gran lacra en nuestros días \cite{OMS} y uno de los aspectos importantes en el tratamiento de este grupo de enfermedades, es la buena disposición y estado mental positivo del enfermo. No cura la enfermedad pero ayuda a superar los duros tratamientos a los que se ven sometidos los enfermos.

Por otro lado, de un tiempo a esta parte las aplicaciones móviles o apps forman parte de nuestros smartphones y por tanto de nuestras vidas. Son tan comunes que para cada ámbito o actividad existe una app que apoya, ayuda, informa o al menos lo pretende con mayor o menor éxito.

Dentro de las apps relacionadas con la salud ninguna o muy pocas de esas tratan sobre esta enfermedad o grupo de enfermedades. Debido a que ambos fuimos tocados de cerca por esta enfermedad y que a la Asociación Española contra el Cáncer le surgió la necesidad decidimos apoyar de la mejor manera que podemos que es escuchando sus necesidades e implementándolas en una app para el sistema operativo móvil Android y aconsejándoles las mejores alternativas en ese sentido. 

Mediante el desarrollo de esta aplicación pretendemos hacer más llevaderos los procesos que ocurren a continuación de pasar por el postraumático tratamiento y cicatrices visibles o no que deja esta enfermedad y devolver la normalidad a esos pacientes que quieren continuar de una manera normal con sus vidas. Haciéndoles más llevaderas algunas de las tareas de obligado cumplimiento 'posterior' que deben realizar estos verdaderos luchadores y supervivientes.
Sin ánimo de ser victimistas, nosotros mismos conocemos de primera mano todo el sufrimiento que se genera, la angustia y el dolor que causa esta lacra y una de las principales causas de muerte en el mundo.

\section{Ámbito de trabajo}

En el desarrollo de este proyecto hay dos ámbitos de trabajo, cercanos, pero ligeramente diferenciados.

El primero de ellos es el de una app de lo que podríamos usuario final, para el uso que el propio usuario crea conveniente dentro de las capacidades de la misma. Es por eso que la usabilidad, el diseño, los tutoriales, ayudas etc deben ir en linea con que el espectro de usuarios potenciales es muy amplio, y que ademas están atravesando un trance complicado.

El segundo ámbito de trabajo se enmarca en la parte servidora, por usuarios de la misma, aunque en este caso podemos presuponer cierto entrenamiento y hábito en el uso de herramientas web, además de que desde la propia AECC se formará a quienes estén manejando esta web y que hemos denominado agentes.

En resumen el uso y ámbito de trabajo donde general es el de uso publico por usuarios no categorizados ni expertos por un lado, y usuario final con entrenamiento y conocimiento por otro.

\section{Objetivos del proyecto}
El objetivo principal del proyecto es servir de apoyo a las acciones emprendidas por la AECC en el marco de la nueva estrategia de identidad digital mediante aplicaciones útiles para cada una de sus iniciativas y las de apoyo general e las acciones de la asociación.

Por tanto uno de los objetivos del proyecto por tanto será el desarrollo de una app móvil denominada "Diario de un Superviviente" que sirva a los intereses de la AECC en la línea de apps en los stores y presencia corporativa e imagen digital. Es por esto que no solamente será el objetivo de este proyecto la realización de la app, sino que la ayuda a la campaña en la web mediante banners, y desarrollo de la propia sección en la Play store, quedando para futuras evoluciones el desarrollo y la inclusión en la tienda del sistema operativo iOS.

En este sentido el otro gran objetivo es desarrollar una parte servidora, en forma de API REST, para realizar sincronizaciones entre dispositivos, recogida de estadísticas, seguimiento por parte del agente asignado, estos últimos puntos además presentados en una web para dicho agente aprovechando la API antes mencionada, quedando también para futuras lineas la implementación de apps que sirvan a los agentes en las tareas que ahora realizan con la web.

\section{Estructura del proyecto}

Para el desarrollo de este aplicativo se seguirán las pautas generales del Agile Manifesto, siguiendo la metodología conocida como Scrum, ya que nos permite desarrollar siempre sobre algo ejecutable y tiene una buena "pelea contra el tiempo" o "timeboxing" ya que al final de cada sprint hay que hacer una retrospectiva sobre lo que ya se ha construido y entregado. Hablar someramente de scrum

\section{Estructura de la memoria}

Esta memoria se distribuye en X capítulos etc etc etc 

\end{document}