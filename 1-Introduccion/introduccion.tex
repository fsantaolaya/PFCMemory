\documentclass[../pfc.tex]{subfiles}

\begin{document}

\section{Antecedentes y Motivación del proyecto}
El cáncer es una de las principales causas de morbilidad y mortalidad en todo el mundo; en 2012 hubo unos 14 millones de nuevos casos y 8,2 millones de muertes relacionadas con el cáncer.1
Se prevé que el número de nuevos casos aumente en aproximadamente un 70% en los próximos 20 años.
En 2012, los cánceres diagnosticados con más frecuencia en el hombre fueron los de pulmón, próstata, colon y recto, estómago e hígado.
En la mujer fueron los de mama, colon y recto, pulmón, cuello uterino y estómago.
Aproximadamente un 30% de las muertes por cáncer son debidas a cinco factores de riesgo conductuales y dietéticos: índice de masa corporal elevado, ingesta reducida de frutas y verduras, falta de actividad física, consumo de tabaco y consumo de alcohol.
El consumo de tabaco es el factor de riesgo más importante, y es la causa más del 20% de las muertes mundiales por cáncer en general, y alrededor del 70% de las muertes mundiales por cáncer de pulmón.
Los cánceres causados por infecciones víricas, tales como las infecciones por virus de las hepatitis B (VHB) y C (VHC) o por papilomavirus humanos (PVH), son responsables de hasta un 20% de las muertes por cáncer en los países de ingresos bajos y medios.1
Más del 60% de los nuevos casos anuales totales del mundo se producen en África, Asia, América Central y Sudamérica. Estas regiones representan el 70% de las muertes por cáncer en el mundo
Se prevé que los casos anuales de cáncer aumentarán de 14 millones en 2012 a 22 millones en las próximas dos décadas .1
«Cáncer» es un término genérico que designa un amplio grupo de enfermedades que pueden afectar a cualquier parte del organismo; también se habla de «tumores malignos» o «neoplasias malignas». Una característica del cáncer es la multiplicación rápida de células anormales que se extienden más allá de sus límites habituales y pueden invadir partes adyacentes del cuerpo o propagarse a otros órganos, proceso conocido como metástasis. Las metástasis son la principal causa de muerte por cáncer.

El problema
El cáncer es la principal causa de muerte a escala mundial. Se le atribuyen 8,2 millones de defunciones ocurridas en todo el mundo en 2012.1 Los principales tipos de cáncer son los siguientes:

pulmonar (1,59 millones de defunciones);
hepático (745 000 defunciones);
gástrico (723 000 defunciones);
colorrectal (694 000) defunciones;
mamario (521 000 defunciones);
cáncer de esófago (400 000 defunciones).
¿Cuál es la causa del cáncer?
El cáncer comienza en una célula. La transformación de una célula normal en tumoral es un proceso multifásico y suele consistir en la progresión de una lesión precancerosa a un tumor maligno. Estas alteraciones son el resultado de la interacción entre los factores genéticos del paciente y tres categorías de agentes externos, a saber:

carcinógenos físicos, como las radiaciones ultravioleta e ionizantes;
carcinógenos químicos, como los asbestos, los componentes del humo de tabaco, las aflatoxinas (contaminantes de los alimentos) o el arsénico (contaminante del agua de bebida);
carcinógenos biológicos, como las infecciones causadas por determinados virus, bacterias o parásitos.
A través de su Centro Internacional de Investigaciones sobre el Cáncer, la OMS mantiene una clasificación de los agentes cancerígenos.

El envejecimiento es otro factor fundamental en la aparición del cáncer. La incidencia de esta enfermedad aumenta muchísimo con la edad, muy probablemente porque se van acumulando factores de riesgo de determinados tipos de cáncer. La acumulación general de factores de riesgo se combina con la tendencia que tienen los mecanismos de reparación celular a perder eficacia con la edad.

Factores de riesgo del cáncer
El consumo de tabaco y alcohol, la dieta malsana y la inactividad física son los principales factores de riesgo de cáncer en todo el mundo. Algunas infecciones crónicas también constituyen factores de riesgo, y son más importantes en los países de ingresos medios y bajos.

Los virus de las hepatitis B (VHB) y C (VHC) y algunos tipos de papilomavirus humanos (PVH) aumentan el riesgo de cáncer de hígado y cuello uterino, respectivamente. La infección por el VIH también aumenta considerablemente el riesgo de algunos cánceres, como los del cuello uterino.

¿Cómo se puede reducir la carga de morbilidad por cáncer?

Se sabe mucho acerca de las causas del cáncer y las intervenciones para prevenirlo y tratarlo. Es posible reducir y controlar el cáncer aplicando estrategias de base científica destinadas a la prevención de la enfermedad así como a la detección temprana y al tratamiento de estos enfermos. Muchos cánceres tienen grandes probabilidades de curarse si se detectan tempranamente y se tratan de forma adecuada.

Modificación y prevención de los riesgos
Más del 30% de las defunciones por cáncer podrían evitarse modificando o evitando los principales factores de riesgo, tales como:

el consumo de tabaco;
el exceso de peso o la obesidad;
las dietas malsanas con un consumo insuficiente de frutas y hortalizas;
la inactividad física;
el consumo de bebidas alcohólicas;
las infecciones por PVH y VHB;
radiaciones ionizantes y no ionizantes;
la contaminación del aire de las ciudades;
el humo generado en la vivienda por la quema de combustibles sólidos.
El consumo de tabaco es el factor de riesgo más importante, y es la causa de aproximadamente un 22% de las muertes mundiales por cáncer en general, y de acerca el 70% de las muertes mundiales por cáncer de pulmón. En muchos países de ingresos bajos, hasta un 20% de las muertes por cáncer son debidas a infecciones por VHB o PVH.

Estrategias de prevención
intensificar la evitación de los factores de riesgo recién enumerados;
vacunar contra los PVH y el VHB;
controlar los riesgos ocupacionales;
reducir la exposición a la radiación no ionizante solar (ultravioleta);
reducir la exposición a la radiación ionizante (ocupacional o pruebas médicas radiológicas).
Detección temprana
La mortalidad por cáncer se puede reducir si los casos se detectan y tratan a tiempo. Las actividades de detección temprana tienen dos componentes:

El diagnóstico temprano
El conocimiento de los síntomas y signos iniciales (en el caso de cánceres como los de la piel, mama, colon y recto, cuello uterino o boca) es fundamental para que se puedan diagnosticar y tratar precozmente. El diagnóstico temprano es especialmente importantes cuando no hay métodos de cribado eficaces o, como ocurre en muchos entornos con escasos recursos, o no se aplican intervenciones de cribado y tratamiento. En ausencia de intervenciones de detección temprana o de cribado y tratamiento, los pacientes son diagnosticados en estadios muy tardíos, cuando yo no son posibles los tratamientos curativos.

El cribado
El cribado tiene por objeto descubrir a los pacientes que presentan anomalías indicativas de un cáncer determinado o de una lesión precancerosa y así poder diagnosticarlos y tratarlos prontamente. Los programas de cribado son especialmente eficaces en relación con tipos de cáncer frecuentes para los cuales existe una prueba de detección costoeficaz, asequible, aceptable y accesible a la mayoría de la población en riesgo.

Estos son algunos ejemplos:

la inspección visual con ácido acético para el cáncer cervicouterino en entornos con pocos recursos;
pruebas de detección de PVH en el caso del cáncer cervicouterino;
el frotis de Papanicolaou para el cáncer cervicouterino en entornos con ingresos medios y altos;
la mamografía para el cáncer de mama en entornos con ingresos altos.
Tratamiento
El diagnóstico correcto del cáncer es esencial para un tratamiento adecuado y eficaz, porque cada tipo de cáncer necesita un tratamiento específico que puede abarcar una o más modalidades, tales como la cirugía, la radioterapia o la quimioterapia. El objetivo principal radica en curar el cáncer o prolongar la vida de forma considerable. Otro objetivo importante consiste en mejorar la calidad de vida del paciente, lo cual se puede lograr con atención paliativa y apoyo psicológico.

Posibilidades de curación de cánceres detectables tempranamente
Algunas de las formas más comunes de cáncer, como el mamario, el cervicouterino, el bucal o el colorrectal, tienen tasas de curación más elevadas cuando se detectan pronto y se tratan correctamente.

Posibilidades de curación de otros cánceres
Algunos tipos de cáncer, a pesar de ser diseminados, como las leucemias y los linfomas en los niños o el seminoma testicular, tienen tasas de curación elevadas si se tratan adecuadamente.

Cuidados paliativos
Como su nombre indica, van dirigidos a aliviar, no a curar, los síntomas del cáncer. Pueden ayudar a los enfermos a vivir más confortablemente; se trata de una necesidad humanitaria urgente para las personas de todo el mundo aquejadas de cáncer u otras enfermedades crónicas mortales. Se necesitan sobre todo en lugares donde hay una gran proporción de enfermos en fase avanzada, que tienen pocas probabilidades de curarse.

Los cuidados paliativos pueden aliviar los problemas físicos, psicosociales y espirituales de más del 90% de los enfermos con cáncer avanzado.

Estrategias de cuidados paliativos
Las estrategias eficaces de salud pública, que abarcan la asistencia comunitaria y en el propio hogar, son esenciales para ofrecer alivio del dolor y cuidados paliativos a los enfermos y a sus familias en los entornos con pocos recursos.

El tratamiento del dolor moderado a intenso causado por el cáncer, que aqueja a más del 80% de los enfermos oncológicos en fase terminal, requiere obligatoriamente una mejora del acceso a la morfina por vía oral.

Respuesta de la OMS
En 2013, la OMS puso en marcha el Plan de Acción Global para la Prevención y el Control de las Enfermedades No Transmisibles 2013-2020 que tiene como objetivo reducir la mortalidad prematura por el 25% de cáncer, enfermedades cardiovasculares, diabetes y enfermedades respiratorias crónicas. Algunas de las metas de aplicación voluntaria son especialmente importantes para la prevención del cáncer, como la que propone reducir el consumo de tabaco en un 30% entre 2014 y 2025.

La OMS y el Centro Internacional de Investigaciones sobre el Cáncer colaboran con otras organizaciones que forman parte del Equipo de Tareas Interinstitucional de las Naciones Unidas sobre la Prevención y el Control de las Enfermedades No Transmisibles y con otros asociados a el fin de:

aumentar el compromiso político con la prevención y el control del cáncer;
coordinar y llevar a cabo investigaciones sobre las causas del cáncer y los mecanismos de la carcinogénesis en el ser humano;
efectuar un seguimiento de la carga de cáncer (como parte de la labor de la Iniciativa Mundial sobre Registros Oncológicos);
elaborar estrategias científicas de prevención y control del cáncer;
generar y divulgar conocimientos para facilitar la aplicación de métodos de control del cáncer basados en datos científicos;
elaborar normas e instrumentos para orientar la planificación y la ejecución de las intervenciones de prevención, detección temprana, tratamiento y atención;
facilitar la formación de amplias redes mundiales, regionales y nacionales de asociados y expertos en el control del cáncer;
fortalecer los sistemas de salud locales y nacionales para que presten servicios asistenciales y curativos a los pacientes con cáncer;
prestar asistencia técnica para la transferencia rápida y eficaz de las prácticas óptimas a los países en desarrollo.

El cáncer es una gran lacra en nuestros días \cite{OMS} y uno de los aspectos importantes en el tratamiento de este grupo de enfermedades, es la buena disposición y estado mental positivo del enfermo. No cura la enfermedad pero ayuda a superar los duros tratamientos a los que se ven sometidos los enfermos.

Por otro lado, de un tiempo a esta parte las aplicaciones móviles o apps forman parte de nuestros smartphones y por tanto de nuestras vidas. Son tan comunes que para cada ámbito o actividad existe una app que apoya, ayuda, informa o al menos lo pretende con mayor o menor éxito.

Dentro de las apps relacionadas con la salud ninguna o muy pocas de esas tratan sobre esta enfermedad o grupo de enfermedades. Debido a que ambos fuimos tocados de cerca por esta enfermedad y que a la Asociación Española contra el Cáncer le surgió la necesidad decidimos apoyar de la mejor manera que podemos que es escuchando sus necesidades e implementándolas en una app para el sistema operativo móvil Android y aconsejándoles las mejores alternativas en ese sentido. 

Mediante el desarrollo de esta aplicación pretendemos hacer más llevaderos los procesos que ocurren a continuación de pasar por el postraumático tratamiento y cicatrices visibles o no que deja esta enfermedad y devolver la normalidad a esos pacientes que quieren continuar de una manera normal con sus vidas. Haciéndoles más llevaderas algunas de las tareas de obligado cumplimiento 'posterior' que deben realizar estos verdaderos luchadores y supervivientes.
Sin ánimo de ser victimistas, nosotros mismos conocemos de primera mano todo el sufrimiento que se genera, la angustia y el dolor que causa esta lacra y una de las principales causas de muerte en el mundo.

Esta aplicación surge como complemento digital a un dossier/iniciativa de la AEEC que pretende servir al enfermo como centro de datos y recolección de información que es el  Diario de un superviviente, en ese sentido esta app pretende ofrecer todo lo que permite el dossier adaptado al mundo digital y aportar algún valor, como pudieran ser filtros sobre los datos, alarmas y notificaciones en las citas, y como modo más experimental el contacto directo con el agente del la propia AECC como si de una app de mensajería instantánea se tratase. 

\section{Ámbito de trabajo}

En el desarrollo de este proyecto hay dos ámbitos de trabajo, cercanos, pero ligeramente diferenciados.

El primero de ellos es el de una app de lo que podríamos usuario final, para el uso que el propio usuario crea conveniente dentro de las capacidades de la misma. Es por eso que la usabilidad, el diseño, los tutoriales, ayudas etc deben ir en linea con que el espectro de usuarios potenciales es muy amplio, y que ademas están atravesando un trance complicado.

El segundo ámbito de trabajo se enmarca en la parte servidora, por usuarios de la misma, aunque en este caso podemos presuponer cierto entrenamiento y hábito en el uso de herramientas web, además de que desde la propia AECC se formará a quienes estén manejando esta web y que hemos denominado agentes.

En resumen el uso y ámbito de trabajo donde general es el de uso publico por usuarios no categorizados ni expertos por un lado, y usuario final con entrenamiento y conocimiento por otro.

\section{Objetivos del proyecto}
El objetivo principal del proyecto es servir de apoyo a las acciones emprendidas por la AECC en el marco de la nueva estrategia de identidad digital mediante aplicaciones útiles para cada una de sus iniciativas y las de apoyo general e las acciones de la asociación, más concreto en este primer paso con la iniciativa de Diario de un Superviviente.

Por tanto uno de los objetivos del proyecto por tanto será el desarrollo de una app móvil denominada "Diario de un Superviviente" que sirva a los intereses de la AECC en la línea de apps en los stores y presencia corporativa e imagen digital. Es por esto que no solamente será el objetivo de este proyecto la realización de la app, sino que la ayuda a la campaña en la web mediante banners, y desarrollo de la propia sección en la Play store, quedando para futuras evoluciones el desarrollo y la inclusión en la tienda del sistema operativo iOS.

En este sentido el otro gran objetivo es desarrollar una parte servidora, en forma de API REST, para realizar sincronizaciones entre dispositivos, recogida de estadísticas, seguimiento por parte del agente asignado, estos últimos puntos además presentados en una web para dicho agente aprovechando la API antes mencionada, quedando también para futuras lineas la implementación de apps que sirvan a los agentes en las tareas que ahora realizan con la web.

\section{Estructura del proyecto}

El proyecto se estructurará en tres partes bien diferenciadas la app móvl, la parte servidora y por último toda la documentación, tanto de esta memoria como los manuales de instalación y usuario del resto de partes. 

\section{Estructura de la memoria}

Esta memoria se distribuye en X capítulos etc etc etc 

\end{document}