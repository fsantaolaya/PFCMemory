\documentclass[../pfc.tex]{subfiles}

\begin{document}

\section{Antecedentes y Motivación del proyecto}
El cáncer es una enfermedad provocada por un grupo de células que se multiplican sin control y de manera autónoma, invadiendo localmente y a distancia otros tejidos. El cáncer es una de las principales causas de mortalidad en todo el mundo; en 2012 hubo unos 14 millones de nuevos casos y 8,2 millones de muertes relacionadas con el cáncer.\\

Se prevé que el número de nuevos casos aumente en aproximadamente un 70\% en los próximos 20 años. En 2012, los cánceres diagnosticados con más frecuencia en el hombre fueron los de pulmón, próstata, colon y recto, estómago e hígado, en la mujer fueron los de mama, colon y recto, pulmón, cuello uterino y estómago.\\

Aproximadamente un 30\% de las muertes por cáncer son debidas a cinco factores de riesgo conductuales y dietéticos: índice de masa corporal elevado, ingesta reducida de frutas y verduras, falta de actividad física, consumo de tabaco y consumo de alcohol.\\

«Cáncer» es un término genérico que designa un amplio grupo de enfermedades que pueden afectar a cualquier parte del organismo; también se habla de «tumores malignos» o «neoplasias malignas». Una característica del cáncer es la multiplicación rápida de células anormales que se extienden más allá de sus límites habituales y pueden invadir partes adyacentes del cuerpo o propagarse a otros órganos, proceso conocido como metástasis. Las metástasis son la principal causa de muerte por cáncer.\\

¿Cuál es la causa del cáncer?\\
El cáncer comienza en una célula en la que se produce la transformación de una célula normal en tumoral.

El envejecimiento es otro factor fundamental en la aparición del cáncer. La incidencia de esta enfermedad aumenta muchísimo con la edad, muy probablemente porque se van acumulando factores de riesgo de determinados tipos de cáncer. La acumulación general de factores de riesgo se combina con la tendencia que tienen los mecanismos de reparación celular a perder eficacia con la edad.\\

Factores de riesgo del cáncer
El consumo de tabaco y alcohol, la dieta malsana y la inactividad física son los principales factores de riesgo de cáncer en todo el mundo. Algunas infecciones crónicas también constituyen factores de riesgo, y son más importantes en los países de ingresos medios y bajos.\\

En resumidas cuentas, el cáncer es una gran lacra en nuestros días \cite{OMS} y uno de los aspectos importantes en el tratamiento de este grupo de enfermedades, es la buena disposición y estado mental positivo del enfermo. No cura la enfermedad pero ayuda a superar los duros tratamientos a los que se ven sometidos los enfermos.\\

Por otro lado, de un tiempo a esta parte las aplicaciones móviles o apps forman parte de nuestros smartphones y por tanto de nuestras vidas. Son tan comunes que para cada ámbito o actividad existe una app que apoya, ayuda, informa o al menos lo pretende con mayor o menor éxito.\\

Dentro de las apps relacionadas con la salud ninguna o muy pocas de esas tratan sobre esta enfermedad o grupo de enfermedades. Debido a que ambos fuimos tocados de cerca por esta enfermedad y que a la Asociación Española contra el Cáncer le surgió la necesidad decidimos apoyar de la mejor manera que podemos que es escuchando sus necesidades e implementándolas en una app para el sistema operativo móvil Android y aconsejándoles las mejores alternativas en ese sentido. \\

Mediante el desarrollo de esta aplicación pretendemos hacer más llevaderos los procesos que ocurren a continuación de pasar por el postraumático tratamiento y cicatrices visibles o no que deja esta enfermedad y devolver la normalidad a esos pacientes que quieren continuar de una manera normal con sus vidas. Haciéndoles más llevaderas algunas de las tareas de obligado cumplimiento 'posterior' que deben realizar estos verdaderos luchadores y supervivientes.\\
Sin ánimo de ser victimistas, nosotros mismos conocemos de primera mano todo el sufrimiento que se genera, la angustia y el dolor que causa esta lacra y una de las principales causas de muerte en el mundo.\\

Esta aplicación surge como complemento digital a un dossier/iniciativa de la AECC que pretende servir al enfermo como centro de datos y recolección de información que es el  Diario de un superviviente, en ese sentido esta app pretende ofrecer todo lo que permite el dossier adaptado al mundo digital y aportar algún valor, como pudieran ser filtros sobre los datos, alarmas y notificaciones en las citas, y como modo más experimental el contacto directo con el agente del la propia AECC como si de una app de mensajería instantánea se tratase. \\


\section{Ámbito de trabajo}

En el desarrollo de este proyecto hay dos ámbitos de trabajo, cercanos, pero ligeramente diferenciados.\\

El primero de ellos es el de una app de lo que podríamos usuario final, para el uso que el propio usuario crea conveniente dentro de las capacidades de la misma. Es por eso que la usabilidad, el diseño, los tutoriales, ayudas etc deben ir en linea con que el espectro de usuarios potenciales es muy amplio, y que ademas están atravesando un trance complicado.\\

El segundo ámbito de trabajo se enmarca en la parte servidora, por usuarios de la misma, aunque en este caso podemos presuponer cierto entrenamiento y hábito en el uso de herramientas web, además de que desde la propia AECC se formará a quienes estén manejando esta web y que hemos denominado agentes.\\

En resumen el uso y ámbito de trabajo donde general es el de uso publico por usuarios no categorizados ni expertos por un lado, y usuario final con entrenamiento y conocimiento por otro.

\section{Objetivos del proyecto}
El objetivo principal del proyecto es servir de apoyo a las acciones emprendidas por la AECC en el marco de la nueva estrategia de identidad digital mediante aplicaciones útiles para cada una de sus iniciativas y las de apoyo general e las acciones de la asociación, más concreto en este primer paso con la iniciativa de Diario de un Superviviente.\\

Por tanto uno de los objetivos del proyecto por tanto será el desarrollo de una app móvil denominada "Diario de un Superviviente" que sirva a los intereses de la AECC en la línea de apps en los stores y presencia corporativa e imagen digital. Es por esto que no solamente será el objetivo de este proyecto la realización de la app, sino que la ayuda a la campaña en la web mediante banners, y desarrollo de la propia sección en la Play store, quedando para futuras evoluciones el desarrollo y la inclusión en la tienda del sistema operativo iOS.\\

En este sentido el otro gran objetivo es desarrollar una parte servidora, en forma de API REST, para realizar sincronizaciones entre dispositivos, recogida de estadísticas, seguimiento por parte del agente asignado, estos últimos puntos además presentados en una web para dicho agente aprovechando la API antes mencionada, quedando también para futuras lineas la implementación de apps que sirvan a los agentes en las tareas que ahora realizan con la web.\\

\section{Estructura del proyecto}

El proyecto se estructurará en tres partes bien diferenciadas la app móvil, la parte servidora y por último toda la documentación, tanto de esta memoria como los manuales de instalación y usuario del resto de partes. 

\section{Estructura de la memoria}

Explicación breve del contenido de cada capítulo:\\

  \textbf{Capítulo 1} contexto general en el que se desenvuelve la aplicación, explicando el problema existente y la solución propuesta para dicho problema. Termina mostrando la estructura del presente documento.\\
  
  \textbf{Capítulo 2} breve explicación de las tecnologías seleccionadas resumen de la plataforma elegida, presentación general del proyecto y justificación del mismo.\\
  
  \textbf{Capítulo 3} desarrollo de la aplicación, planificación, plazos, técnicas utilizadas para llevar a cabo las funcionalidades de la aplicación.\\
  
  \textbf{Capítulo 4} fase de análisis donde podrá encontrar aspectos clave para el buen desarrollo del producto como una introducción a la aplicación y los estudios sobre la arquitectura utilizada, casos de uso y requisitos, diagrama de clases, entidad-relación.\\
  
  \textbf{Capítulo 5} diseño, determina el comportamiento esperado de la aplicación mediante una serie de diagramas de secuencia y un prototipado de bajo nivel.\\
  
  \textbf{Capítulo 6} construcción de la aplicación, como se ha conseguido llevar a cabo la implementación de las funcionalidades de la aplicación. Además incluye las pruebas realizadas para determinar el buen funcionamiento de la misma.\\
  
  \textbf{Capítulo 7} presenta las conclusiones obtenidas una vez finalizado el desarrollo del proyecto, y una serie de trabajos futuros que podrían desarrollarse como mejoras de la aplicación.\\
  
  \textbf{Capítulo 8} bibliografía y referencias web, que incluyen toda documentación consultada para la elaboración del proyecto.\\ 
  
  \textbf{Anexo I} contiene el manual de usuario, así como el manual de instalación para aquellos usuarios que no estén tan familiarizados con la plataforma Android.\\

 

\end{document}