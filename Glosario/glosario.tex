\documentclass[../pfc.tex]{subfiles}

\begin{document}
	
	\textbf{Android}
	
	Android es un sistema operativo basado en el núcleo Linux. Fue diseñado principalmente para dispositivos móviles con pantalla táctil, como teléfonos inteligentes o tablets; y también para relojes inteligentes, televisores y automóviles. Inicialmente fue desarrollado por Android Inc., empresa que Google respaldó económicamente y más tarde, en 2005, compró. Como curiosidad: Tanto el nombre Android (androide en español) como Nexus hacen alusión a la novela de Philip K. Dick ¿Sueñan los androides con ovejas eléctricas?, que posteriormente fue adaptada al cine como Blade Runner. Tanto el libro como la película se centran en un grupo de androides llamados replicantes del modelo Nexus-6.\\
	
	\textbf{Android Studio}
	
	Android Studio es un entorno de desarrollo integrado (IDE) para la plataforma Android. Fue anunciado por Ellie Powers el 16 de mayo de 2013. Android Studio esta disponible para desarrolladores para probarlo gratuitamente. Basado en IntelliJ IDEA de JetBrains, está diseñado específicamente para desarrollar para Android. Esta disponible para descargar para Windows, Mac OS X y Linux.\\
	
	\textbf{BBDD.}
	
	Una base de datos o banco de datos es un conjunto de datos pertenecientes a un mismo
	contexto y almacenados sistemáticamente para su posterior uso. Una base de datos es un “almacén”
	que nos permite guardar grandes cantidades de información de forma organizada para que luego
	podamos encontrar y utilizar fácilmente.\\
	
	\textbf{DAO.}
	
	Los Objetos de Acceso a Datos son un Patrón de Diseño y considerados una buena práctica.
	La ventaja de usar objetos de acceso a datos es que cualquier objeto de negocio (aquel que contiene
	detalles específicos de operación o aplicación) no requiere conocimiento directo del destino final de
	la información que manipula.\\
	
	\textbf{Dropbox.}
	
	Se trata de una herramienta de sincronización de archivos a través de un disco duro o
	directorio virtual. Permite disponer de un directorio de archivos de forma remota y accesible desde
	cualquier ordenador. Es decir, crea una carpeta en nuestro ordenador y realiza una copia a través
	de Internet de todos los archivos que depositemos en ella. Se ocupa de mantener la copia de
	nuestros archivos siempre sincronizada.\\
	
	\textbf{IDE (Integrated Development Environment).}
	
	Un entorno de desarrollo integrado es un programa
	informático compuesto por un conjunto de herramientas de programación. Puede dedicarse en
	exclusiva a un sólo lenguaje de programación o bien, poder utilizarse para varios. Consiste en un
	editor de código, un compilador, un depurador y un constructor de interfaz gráfica (GUI). Los IDEs
	pueden ser aplicaciones por sí solas o pueden ser parte de aplicaciones existentes.\\
	
	\textbf{Java.}
	
	Es un lenguaje de programación orientado a objetos y la primera plataforma informática
	creada por Sun Microsystems en 1995. Es la tecnología subyacente que permite el uso de
	programas punteros, como herramientas, juegos y aplicaciones de negocios. Tiene como principal
	característica ser un lenguaje independiente de la plataforma.\\
	
	\textbf{MVC.}
	
	Modelo Vista Controlador, es un patrón de arquitectura de software que separa los datos de
	la aplicación, la interfaz de usuario, y la lógica de control en tres componentes distintos. Se ve
	frecuentemente en aplicaciones web, donde la vista es la página HTML y el código que provee de
	datos dinámicos a la página. El modelo es el Sistema de Gestión de Base de Datos y la Lógica de
	negocio, y el controlador es el responsable de recibir los eventos de entrada desde la vista.\\
	
	\textbf{MVP.}
	
	Modelo Vista Presentador, es un patrón de arquitectura de software que se parece al MVC pero debido al gran acoplamiento existente en android y que el controlador y la vista puedan acabar fusionados en la Activity es más usado que el anterior (seguir).\\
	
	\textbf{MySQL.}
	
	MySQL es un sistema de administración de bases de datos para bases de datos
	relacionales. No es más que una aplicación que permite gestionar archivos llamados de bases de
	datos. Utilizado para almacenar todos los datos de interés de la aplicación y datos referentes de las
	simulaciones realizadas por los usuarios.\\
	
	\textbf{StarUML.}
	
	Es una herramienta para el modelamiento de software basado en los estándares UML
	(Unified Modeling Language) y MDA (Model Driven Arquitectura). Permite definir elementos
	propios para los diagramas, que no necesariamente tienen que pertenecer al estándar de UML.\\
	
	\textbf{SGBD (Sistema Gestor de Bases de Datos).}
	
	En inglés Data Base Management System (DBMS); es
	un tipo de software muy específico, dedicado a servir de interfaz entre la base de datos y el usuario
	y las aplicaciones que la utilizan. Se compone de un lenguaje de definición de datos, de un lenguaje
	de manipulación de datos y de un lenguaje de consulta.\\
	
	\textbf{USB (Universal Serial Bus).}
	
	Estándar de comunicaciones serie de alta velocidad.
	Vensim. Es una herramienta de modelaje que permite conceptualizar, documentar, simular,
	analizar y optimizar modelos de dinámica de sistemas. Vensim provee una forma simple y flexible
	de construir modelos de simulación, sean lazos causales o diagramas de stock y de flujo.\\
	
	\textbf{USB (Pair Programming.}
	
	Tecnica (seguir).\\
	
	\textbf{TDD.}
	
	Tecnica.\\
	
	\textbf{BDD.}
	
	Tecnica.\\
	
	\textbf{Code Review.}
	
	tecnica.\\
	
	\textbf{Mockup.}
	
	Doble.\\
	
	\textbf{Code Smell.}
	
	mal codigo.\\
\end{document}