\documentclass[../pfc.tex]{subfiles}

\begin{document}
	
	\textbf{AECC.}
	La Asociación Española Contra el Cáncer (aecc) es una entidad privada, benéfica y sin ánimo de lucro, declarada de interés público. Fue fundada y aprobada por Orden Ministerial en el año 1953.
	Los objetivos de la asociación han ido variando desde su fundación hasta el momento actual, intentando siempre dar una respuesta a las necesidades que se han ido planteando en el ámbito de los enfermos de cáncer y sus familias.\\*
	
	\textbf{Android}\\*
	Android es un sistema operativo basado en el núcleo Linux. Fue diseñado principalmente para dispositivos móviles con pantalla táctil, como teléfonos inteligentes o tablets; y también para relojes inteligentes, televisores y automóviles. Inicialmente fue desarrollado por Android Inc., empresa que Google respaldó económicamente y más tarde, en 2005, compró. Como curiosidad: Tanto el nombre Android (androide en español) como Nexus hacen alusión a la novela de Philip K. Dick ¿Sueñan los androides con ovejas eléctricas?, que posteriormente fue adaptada al cine como Blade Runner. Tanto el libro como la película se centran en un grupo de androides llamados replicantes del modelo Nexus-6.\\*
	
	\textbf{Android Studio}\\*
	Android Studio es un entorno de desarrollo integrado (IDE) para la plataforma Android. Fue anunciado por Ellie Powers el 16 de mayo de 2013. Android Studio esta disponible para desarrolladores para probarlo gratuitamente. Basado en IntelliJ IDEA de JetBrains, está diseñado específicamente para desarrollar para Android. Esta disponible para descargar para Windows, Mac OS X y Linux.\\*
	
	\textbf{App.}
	En informática, una aplicación es un tipo de programa informático diseñado como herramienta para permitir a un usuario realizar uno o diversos tipos de trabajos. Esto lo diferencia principalmente de otros tipos de programas, como los sistemas operativos (que hacen funcionar la computadora), los utilitarios (que realizan tareas de mantenimiento o de uso general), y los lenguajes de programación (para crear programas informáticos).\\*
	
	\textbf{Arquitectura del software.}
	Es el diseño de más alto nivel de la estructura de un sistema, la arquitectura del software es la estructura jerarquica a traves de la cual se relacionan los modulos del software que reunidos por completo forman nuestra aplicación.\\*
	
	\textbf{Backend.}
	De forma general, back-end hace referencia al estado final de un proceso. Contrasta con front-end, que se refiere al estado inicial de un proceso. 
	La idea general es que el front-end es responsable de recoger entradas de los usuarios, y ser procesadas de tal manera que cumplan las especificaciones para que el back-end pueda usarlas. La conexión entre front-end y el back-end es un tipo de interfaz.\\*
	
	\textbf{BBDD.}
	Una base de datos o banco de datos es un conjunto de datos pertenecientes a un mismo contexto y almacenados sistemáticamente para su posterior uso. Una base de datos es un “almacén” que nos permite guardar grandes cantidades de información de forma organizada para que luego podamos encontrar y utilizar fácilmente.\\*
	
	\textbf{BDD.}
	El Desarrollo Guiado por el Comportamiento o BDD es un proceso que amplia las ideas de TDD y las combina con otras ideas de diseño de software y análisis de negocio para proporcionar un proceso ( y una serie de herramientas ) a los desarrolladores, con la intención de mejorar el desarrollo del software, BDD se basa en TDD formalizando las mejores practicas de TDD, clarificando cuales son y haciendo énfasis en ellas.\\*
	
	\textbf{Cáncer.}
	El cáncer es el nombre común que recibe un conjunto de enfermedades relacionadas en las que se observa un proceso descontrolado en la división de las células del cuerpo. Puede comenzar de manera localizada y diseminarse a otros tejidos circundantes. En general conduce a la muerte del paciente si este no recibe tratamiento adecuado. Se conocen más de 200 tipos diferentes de cáncer. Los más comunes son: de piel, pulmón, mama y colorrectal.\\*
	
	\textbf{Clean Architecture}\\*
	Arquitectura de software cuyo principal objetivo es la separacion de requerimientos o 'concerns' en capas, de tal manera que se produzca el mínimo acoplamiento entre ellas.\\*
	
	\textbf{Clean Code}
	Se refiere a los métodos y recomendaciones que hacen que nuestro 'código' sea más limpio. Todas estas buenas prácticas sobre cómo escribir código limpio se deben ejecitar de forma constante para adquirir buenos hábitos y ser capaces de hacerlo de forma natural.\\* 
	
	\textbf{Code Retreat.}
	Coderetreat es una práctica intensiva de un día de duración, centrándose en los fundamentos de desarrollo de software y diseño, proporcionando a los desarrolladores la oportunidad de tomar parte en la práctica enfocada lejos de las presiones del "Getting Things Done" y haciendo del mismo un método práctico para la mejora de habilidades.\\*
	
	\textbf{Code Review.}
	Code review o revisión de código es el examen sistemático (a menudo conocido como peer review) del código fuente ya sea por nosotros mismos o por otros compañeros.
	Su objetivo es encontrar y corregir errores pasados por alto en la fase inicial de desarrollo, mejorando tanto la calidad global de software como las habilidades de los desarrolladores.\\*
	
	\textbf{Code Smell.}
	Es cualquier síntoma en el código fuente de un programa que posiblemente indica un problema más profundo. Los code smells usualmente no son un bug de programación (errores), no son técnicamente incorrectos y en realidad no impiden que el programa funcione correctamente. En cambio, indican deficiencias en el diseño que puede ralentizar el desarrollo o aumentan el riesgo de errores o fallos en el futuro.\\*
	
	\textbf{Daily meeting.}
	Es la reunión diaria de sincronización del equipo (Scrum daily meeting).
	El objetivo de esta reunión es facilitar la transferencia de información y la colaboración entre los miembros del equipo para aumentar su productividad, al poner de manifiesto puntos en que se pueden ayudar unos a otros.\\*
	
	\textbf{DAO.}
	Los Objetos de Acceso a Datos son un Patrón de Diseño y considerados una buena práctica.
	La ventaja de usar objetos de acceso a datos es que cualquier objeto de negocio (aquel que contiene detalles específicos de operación o aplicación) no requiere conocimiento directo del destino final de la información que manipula.\\*
	
	\textbf{Deuda técnica.}
	La deuda técnica es un eufemismo tecnológico que hace referencia a las consecuencias de un desarrollo apresurado de software o un despliegue descuidado de hardware, puede entenderse perfectamente como chapuza.\\* 
	
	\textbf{Diario de salud para supervivientes de Cáncer.}
	Iniciativa de la Asociación española contra el cancér con el propósito de ayudar a los supervivientes de cáncer a afrontar la nueva etapa de su enfermedad.\\*
	
	\textbf{DPI.}
	Es una unidad de medida para resoluciones de impresión, concretamente, el número de puntos individuales de tinta que una impresora o tóner puede producir en un espacio lineal de una pulgada.
	El numero habitual de dpi para la yema de un dedo es 48.\\*
	
	\textbf{Drawer menú.}
	Menú de navegación con desplazamiento lateral que forma parte de la interfaz de usuario y que ayuda a 'navegar' entre las diferentes actividades que forman la aplicación.\\*
	
	\textbf{Dropbox.}
	Se trata de una herramienta de sincronización de archivos a través de un disco duro o
	directorio virtual. Permite disponer de un directorio de archivos de forma remota y accesible desde
	cualquier ordenador. Es decir, crea una carpeta en nuestro ordenador y realiza una copia a través
	de Internet de todos los archivos que depositemos en ella. Se ocupa de mantener la copia de
	nuestros archivos siempre sincronizada.\\*
	
	\textbf{Git.}
	Es un software de control de versiones diseñado por Linus Torvalds, pensando en la eficiencia y la confiabilidad del mantenimiento de versiones de aplicaciones cuando éstas tienen un gran número de archivos de código fuente. \\*
	
	\textbf{GitHub.}
	Es una plataforma de desarrollo colaborativo para alojar proyectos utilizando el sistema de control de versiones Git.
	El código se almacena de forma pública, aunque también se puede hacer de forma privada, creando una cuenta de pago.\\*
	
	\textbf{Google Play store.}
	Es una plataforma de distribución digital de aplicaciones móviles para los dispositivos con sistema operativo Android, así como una tienda en línea desarrollada y operada por Google. Esta plataforma permite a los usuarios navegar y descargar aplicaciones (desarrolladas mediante Android SDK), juegos, música, libros, revistas y películas.\\*
	
	\textbf{Historía de usuario.}
	Una historia de usuario es una representación de un requisito de software escrito en una o dos frases utilizando el lenguaje común del usuario. Las historias de usuario son utilizadas en las metodologías de desarrollo ágiles para la especificación de requisitos (acompañadas de las discusiones con los usuarios y las pruebas de validación). Cada historia de usuario debe ser limitada, ésta debería poderse escribir sobre una nota adhesiva pequeña. Dentro de la metodología XP las historias de usuario deben ser escritas por los clientes.\\*
	
	\textbf{Historía épica.}
	Una historia épica no es más que un nivel de agrupación por encima de las historias de usuario que permite clasificar las mismas por funcionalidades, módulos, subsistemas, etc… 
	Las epicas se suelen referir a funcionalidades completas que deberán ser divididas en varias historias de usuario.\\*
	
	\textbf{IDE (Integrated Development Environment).}
	Un entorno de desarrollo integrado es un programa
	informático compuesto por un conjunto de herramientas de programación. Puede dedicarse en
	exclusiva a un sólo lenguaje de programación o bien, poder utilizarse para varios. Consiste en un
	editor de código, un compilador, un depurador y un constructor de interfaz gráfica (GUI). Los IDEs
	pueden ser aplicaciones por sí solas o pueden ser parte de aplicaciones existentes.\\*
	
	\textbf{iOS.}
	iOS es un sistema operativo móvil de la multinacional Apple Inc, iOS se deriva de OS X, que a su vez está basado en Darwin BSD, y por lo tanto es un sistema operativo Tipo Unix. Originalmente desarrollado para el iPhone (iPhone OS), después se ha usado en dispositivos como el iPod touch y el iPad. \\*
	
	\textbf{Java.}
	Es un lenguaje de programación orientado a objetos y la primera plataforma informática
	creada por Sun Microsystems en 1995. Es la tecnología subyacente que permite el uso de
	programas punteros, como herramientas, juegos y aplicaciones de negocios. Tiene como principal
	característica ser un lenguaje independiente de la plataforma.\\*
	
	\textbf{Kata.}
	Las katas y dojos son unos ejercicios que se realizan para practicar, son problemas sencillos de los que se conoce la solución pero lo importante no es resolverlos sino aplicar las lecciones aprendidas y mejorar nuestras habilidades de programación que posteriormente usemos en los proyectos que trabajamos.\\*
	
	\textbf{Koan.}
	Koan es un problema que el maestro plantea al alumno para comprobar sus progresos.\\*
	
	\textbf{Material design.}
	Material design es una normativa de diseño enfocado en la visualización del sistema operativo Android, pero también en la web y en cualquier plataforma.  Material se trata de un diseño más limpio, en el que predominan animaciones y transiciones de respuesta, el relleno y los efectos de profundidad tales como la iluminación y las sombras.\\*
	
	\textbf{Metástasis.}
	Cuando el cáncer se propaga desde la parte del cuerpo donde comenzó (sitio primario) a otras partes del cuerpo se le llama metástasis. La metástasis puede ocurrir cuando las células se desprenden de un tumor canceroso y se desplazan a otras áreas del cuerpo a través del torrente sanguíneo o los vasos linfáticos. (Los vasos linfáticos se parecen mucho a los vasos sanguíneos con la diferencia que transportan un líquido claro llamado linfa de regreso al corazón). Las células cancerosas que se trasladan a través de los vasos sanguíneos o linfáticos se pueden propagar a otros órganos o tejidos en partes distantes del cuerpo.\cite{metastasis}\\*\
	
	\textbf{Mockup.}
	Un mockup, mock-up, o maqueta es un modelo a escala o tamaño real de un diseño o un dispositivo, utilizado para la demostración, evaluación del diseño, promoción, y para otros fines. Un mockup es un prototipo si proporciona al menos una parte de la funcionalidad de un sistema y permite pruebas del diseño.
	Los mockups son utilizados por los diseñadores principalmente para la adquisición de comentarios por parte de los usuarios. Los mock-ups abordan la idea capturada en la ingeniería popular: 'Usted puede arreglarlo ahora en el dibujo con una goma de borrar o más tarde en la obra con un martillo'.\\*
	
	\textbf{Movilidad.}
	Se entiende por movilidad en el ámbito de nuestro proyecto como cualquier aplicacion diseñada para dispositivos móviles.\\*
	
	\textbf{MVC.}
	Modelo Vista Controlador, es un patrón de arquitectura de software que separa los datos de la aplicación, la interfaz de usuario, y la lógica de control en tres componentes distintos. Se ve frecuentemente en aplicaciones web, donde la vista es la página HTML y el código que provee de datos dinámicos a la página. El modelo es el Sistema de Gestión de Base de Datos y la Lógica de negocio, y el controlador es el responsable de recibir los eventos de entrada desde la vista.\\*
	
	\textbf{MVP.}
	Modelo Vista Presentador, es un patrón de arquitectura de software que se parece al MVC pero debido al gran acoplamiento existente en android y que el controlador y la vista puedan acabar fusionados en la Activity es más usado que el anterior (seguir).\\*
	
	\textbf{MySQL.}
	MySQL es un sistema de administración de bases de datos para bases de datos relacionales. No es más que una aplicación que permite gestionar archivos llamados de bases de datos. Utilizado para almacenar todos los datos de interés de la aplicación y datos referentes de las simulaciones realizadas por los usuarios.\\*
	
	\textbf{Neoplasia.}
	Formación anormal en alguna parte del cuerpo de un tejido nuevo de carácter tumoral, benigno o maligno.\\*
	
	\textbf{Pair Programming.}
	Se refiere a Programación en Pareja, esta requiere que dos programadores participen en un esfuerzo combinado de desarrollo en un sitio de trabajo, cada miembro realiza una acción que el otro no está haciendo actualmente: Mientras que uno codifica las pruebas de unidades el otro piensa en la clase que satisfará la prueba.\\*
	
	\textbf{Posología.}
	Parte de la farmacología que trata de las dosis en que deben administrarse los medicamentos.\\*
	
	\textbf{Product backlog.}
	Es la lista de objetivos/requisitos priorizada, representa la visión y expectativas del cliente (o PO) respecto a los objetivos y entregas del producto o proyecto. \\*
	
	\textbf{Product Owner.}
	El product owner o PO, representa al cliente (puede ser interno o externo a la organización) y asume sus responsabilidades de cara al equipo, representa a todas las personas interesadas en los resultados del proyecto (desde agentes de la AECC a usuarios finales de la aplicación).\\*

	\textbf{Requisito.}
	En la ingeniería de sistemas, un requisito es una necesidad documentada sobre el contenido, forma o funcionalidad de un producto o servicio. En la ingeniería clásica, los requisitos se utilizan como datos de entrada en la etapa de diseño del producto, en metodologías agiles, se utilizan historias de usuario, sin embargo estas no son equivaklesntes.\\*
	
	\textbf{Retrospectiva.}
	Se trata de una reunión que se realiza al finalizar el sprint actual, después de la demo con el cliente y tiene el objetivo de mejorar de manera continua la productividad y la calidad del producto que está desarrollando.
	El equipo analiza cómo ha sido su manera de trabajar durante la iteración, por qué está consiguiendo o no los objetivos a que se comprometió al inicio de la iteración y por qué el incremento de producto que acaba de demostrar al cliente era lo que él esperaba o no\\*	
	
	\textbf{ROI retorno de la inversión.}
	El retorno sobre la inversión (RSI o ROI) es una razón financiera que compara el beneficio o la utilidad obtenida en relación a la inversión realizada, es decir, 'representa una herramienta para analizar el rendimiento que la empresa tiene desde el punto de vista financiero'.\\*	
	
	\textbf{SCRUM.}
	Scrum es un proceso en el que se aplican de manera regular un conjunto de buenas prácticas para trabajar colaborativamente, en equipo, y obtener el mejor resultado posible de un proyecto.\\*
		
	\textbf{SCRUM MASTER.}	
	También llamado facilitador, Vela por que todos los participantes del proyecto sigan los valores y principios ágiles, las reglas y proceso de Scrum y guía la colaboración intraequipo y con el cliente de manera que las sinergias sean máximas.\\* 
	
	\textbf{SGBD (Sistema Gestor de Bases de Datos).}
	En inglés Data Base Management System (DBMS); es un tipo de software muy específico, dedicado a servir de interfaz entre la base de datos y el usuario y las aplicaciones que la utilizan. Se compone de un lenguaje de definición de datos, de un lenguaje de manipulación de datos y de un lenguaje de consulta.\\*
	
	\textbf{Sintomatología.}
	Conjunto de síntomas que son característicos de una enfermedad determinada o que se presentan en un enfermo.
	También se refiere a la parte de la medicina que estudia los síntomas de las enfermedades.\\*
	
	\textbf{SPRINT.}
	Un sprint (o iteración) es la unidad básica de desarrollo de scrum, se limita a una duración específica.
	La duración se fija de antemano por cada sprint y normalmente es entre una semana y un mes, a dos semanas.\\*
	
	\textbf{SPRINT PLANNING}\\*
	Con sprint planning, nos referimos a la planificación de las funcionalidades que se deben abordar para esa iteración o sprint, las tareas que entran en un sprint se fijan de acuerdo a la priorización de las mismas. 
	
	\textbf{StarUML.}
	Es una herramienta para el modelamiento de software basado en los estándares UML (Unified Modeling Language) y MDA (Model Driven Arquitectura). Permite definir elementos propios para los diagramas, que no necesariamente tienen que pertenecer al estándar de UML.\\*
	
	\textbf{SVN.}
	Es una herramienta de control de versiones open source basada en un repositorio cuyo funcionamiento se asemeja enormemente al de un sistema de ficheros. Es software libre bajo una licencia de tipo Apache/BSD.\\*
	
	\textbf{TDD.}
	 Es una práctica de ingeniería de software que involucra otras dos prácticas: Escribir las pruebas primero (Test First Development) y Refactorización (Refactoring). Para escribir las pruebas generalmente se utilizan las pruebas unitarias (unit test en inglés). En primer lugar, se escribe una prueba y se verifica que las pruebas fallan. A continuación, se implementa el código que hace que la prueba pase satisfactoriamente y seguidamente se refactoriza el código escrito. El propósito del desarrollo guiado por pruebas es lograr un código limpio que funcione. La idea es que los requisitos sean traducidos a pruebas, de este modo, cuando las pruebas pasen se garantizará que el software cumple con los requisitos que se han establecido.\\*
	
	\textbf{Testing.}
	Las pruebas de software es una investigación llevada a cabo para facilitar a los interesados información sobre la calidad del producto o servicio que se está probando.Otra definición: investigación técnica de un producto bajo prueba con el fin de brindar información relativa a la calidad del software, a los diferentes actores involucrados en un proyecto.\\*
	
	\textbf{Time to market.}
	Es el tiempo que que se tarda desde que un producto es concebido hasta que está en el mercado.\\*
	
	\textbf{UI.}
	La interfaz de usuario es el medio con que el usuario puede comunicarse con una máquina, un equipo o una computadora, y comprende todos los puntos de contacto entre el usuario y el equipo. Normalmente suelen ser fáciles de entender y fáciles de accionar (aunque en el ámbito de la informática es preferible referirse a que suelen ser «amigables e intuitivos» pues es muy complejo y subjetivo decir que algo es «fácil»).\\*

	\textbf{USB (Universal Serial Bus).}
	Estándar de comunicaciones serie de alta velocidad. \\*
	
	\textbf{UX.}
	Experiencia de usuario (UX) es un término que mide el nivel de satisfacción total de usuarios cuando utilizan tu producto o sistema.\\* 

	\textbf{XP (extreme programming).}
	 Es una metodología de desarrollo de la ingeniería de software formulada por Kent Beck.
	 Es el más destacado de los procesos ágiles de desarrollo de software, la programación extrema se diferencia de las metodologías tradicionales principalmente en que pone más énfasis en la adaptabilidad que en la previsibilidad. Los defensores de la XP consideran que los cambios de requisitos sobre la marcha son un aspecto natural, inevitable e incluso deseable del desarrollo de proyectos. Creen que ser capaz de adaptarse a los cambios de requisitos en cualquier punto de la vida del proyecto es una aproximación mejor y más realista que intentar definir todos los requisitos al comienzo del proyecto e invertir esfuerzos después en controlar los cambios en los requisitos.\\*
	
\end{document}